\documentclass[12pt,twoside,a4paper]{article}
\usepackage[brazil]{babel}
\usepackage[utf8]{inputenc}
\usepackage[T1]{fontenc}
\usepackage{timbre-ic}
\usepackage{booktabs}
\usepackage[table]{xcolor}
\usepackage{url}
\usepackage{array}

\begin{document}

\vskip 15mm

\begin{center} 
\textbf{Projeto da Disciplina MO447 - Análise Forense de Impressoras}

\end{center}

\vskip 5mm

\textbf{Aluno:} Adriano Ricardo Ruggero

\textbf{Aluno:} Marião Mestre

\textbf{Aluno:} Gabriel, o Pensador

\textbf{Aluno:} Maurício é de Matar

\vskip 20mm

\begin{abstract}

Aqui a gente coloca o resumo...

\end{abstract}

% resetando configs de layout
\newpage
\pagestyle{plain}
\headheight 0.0cm
\headsep 0.0cm
\footskip 2.2cm

\section{Introdução e justificativa}
\label{sec:introduction}

Aqui nós damos uma pequena introdução no pessoal...

\section{Trabalhos relacionados}
\label{sec:related}

Apresentação de trabalhos relacionados (bibliografia pesquisada).	O \textit{teacher} just gave us two... \textit{For instance...}, no trabalho apresentando por José da Silva, em \cite{Bulan} é apresentada a historieta de João e o pé de feijão. Em \cite{Kee}, sabemos de Maria enfrentando a bruxa malvada, como o autor quer que acreditemos. 


\section{Objetivos}
\label{sec:objects}

Qual o nosso objetivo? Viver \textit{forever}?

\section{Plano de trabalho}
\label{sec:plan}

Rapaz, o plano é o seguinte:
 
\section{Materiais e métodos}
\label{sec:materials}

Cumé qui nóis fez?	

\section{Conclusões}
\label{sec:conclusions}

Assim sendo, concluímos que blá, blá, blá...

\bibliography{Biblio}{}
\bibliographystyle{acm}

Aqui aparecerão as referências geradas au-to-ma-ti-ca-men-te!!!! (Ohhhhh!!!)

\end{document}
