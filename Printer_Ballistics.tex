%%% PREAMBLE - Do not touch %%%%%%%%%%%%%%%%%%%%%%%%%%%%%%%%%%%%%%%%%%%%%%%%%%%%%%
\documentclass[10pt,twocolumn,letterpaper]{article}
%%\usepackage[ansinew]{inputenc}
\usepackage[latin1,utf8]{inputenc}
\usepackage[portuges,brazil,english]{babel}
\usepackage{model}
\usepackage{times}
\usepackage{epsfig}
\usepackage{graphicx}
\usepackage{amsmath}
\usepackage{amssymb}
\usepackage{color}
\usepackage[pagebackref=true,breaklinks=true,letterpaper=true,colorlinks,bookmarks=false]{hyperref}
%  ABACO -- Conjunto de macros para desenhar o 'abaco

%  Desenho original de Hans Liesenberg

%  Macros de Tomasz Kowaltowski

%  DCC -- IMECC -- UNICAMP

%  Mar,co de 1988  --  Vers~ao 1.0

% Ajustado para LaTeX da SUN -- Mar,co de 1991

% ---------------------------------------------------------

%  Chamada:   \ABACO{d1}{d2}{d3}{d4}{esc}
%             com:  di's -- os quatro d'igitos;
%	           esc  -- fator de escala

% ---------------------------------------------------------

%  DEFINI,C~OES AUXILIARES

% ---------------------------------------------------------


%  Forma o d'igito pequeno (0 ou 1)

\newcommand{\ABACODP}[1]{%
%
\thicklines
%    
\begin{picture}(8,0)
    \ifcase#1{   %  caso 0
       \put(0,0)    {\line(1,0){4}}
       \multiput(5,0)(2,0){2}{\oval(2,4)}}
    \or{         %  caso 1
       \put(2,0)    {\line(1,0){4}}
       \multiput(1,0)(6,0){2}{\oval(2,4)}}
    \fi
\end{picture}
    } % \ABACODP

% Forma o d'igito grande (0 a 4)

\newcommand{\ABACODG}[1]{%
%
\thicklines
%    
\begin{picture}(14,0)
    \ifcase#1{   % caso 0
       \multiput(1,0)(2,0){5}{\oval(2,4)}}
       \put(10,0)   {\line(1,0){4}}
    \or{         % caso 1
       \multiput(1,0)(2,0){4}{\oval(2,4)}}
       \put(8,0)   {\line(1,0){4}}
       \put(13,0)   {\oval(2,4)}
    \or{         % caso 2
       \multiput(1,0)(2,0){3}{\oval(2,4)}
       \put(6,0)   {\line(1,0){4}}
       \multiput(11,0)(2,0){2}{\oval(2,4)}}
    \or{         % caso 3
       \multiput(1,0)(2,0){2}{\oval(2,4)}
       \put(4,0)   {\line(1,0){4}}
       \multiput(9,0)(2,0){3}{\oval(2,4)}}
    \or{         % caso 4
       \put(1,0)  {\oval(2,4)}}
       \put(2,0)   {\line(1,0){4}}
       \multiput(7,0)(2,0){4}{\oval(2,4)}
    \fi
\end{picture}
    } % \ABACODG
       
% Forma um d'igito (0 a 9)

\newcommand{\ABACOD}[1]{%
%
    \ifnum#1>9
       \errmessage{#1: Argumento invalido para ABACO}
    \fi
    \ifnum#1<0
       \errmessage{#1: Argumento invalido para ABACO}
    \fi
%
\begin{picture}(24,0)
%    
    \ifnum#1<5
       \put(16,0) {\ABACODP{0}}
    \else   
       \put(16,0) {\ABACODP{1}}
    \fi
%    
    \ifnum#1<5
       \put(0,0)  {\ABACODG{#1}}
    \else
       \ifcase#1\or \or \or \or
          \or  \put(0,0)  {\ABACODG{0}}
          \or  \put(0,0)  {\ABACODG{1}}
          \or  \put(0,0)  {\ABACODG{2}}
          \or  \put(0,0)  {\ABACODG{3}}
          \or  \put(0,0)  {\ABACODG{4}}
       \fi
    \fi   
\end{picture}
    } % \ABACOD
    
% -------------------------------------------------

%  DEFINI,C~AO PRINCIPAL
    
\newcommand{\ABACO}[5]{%
    \setlength{\unitlength}{#5mm}
%
    \thinlines
%   
\begin{picture}(28,25)
%   
% moldura
%
% externa
%
        \put(0,0)            {\line(0,1){25}}
        \put(0,0)            {\line(1,0){28}}
        \put(28,0)           {\line(0,1){25}}
        \put(0,25)           {\line(1,0){28}}
% interna
        \put(2,2)            {\line(0,1){21}}
	\put(26,2)           {\line(0,1){21}}
	\put(16,2)           {\line(0,1){21}}
	\put(18,2)           {\line(0,1){21}}
	\put(2,2)            {\line(1,0){14}}
	\put(16,2)           {\line(1,-1){1}}
	\put(17,1)           {\line(1,1){1}}
	\put(18,2)           {\line(1,0){8}}
	\put(2,23)           {\line(1,0){14}}
	\put(16,23)          {\line(1,1){1}}
	\put(17,24)          {\line(1,-1){1}}
	\put(18,23)          {\line(1,0){8}}
	\put(0,0)            {\line(1,1){2}}
	\put(0,25)           {\line(1,-1){2}}
	\put(28,0)           {\line(-1,1){2}}
	\put(28,25)          {\line(-1,-1){2}}
%
%   
% d'igitos
%
%   
       \put(2,20)  {\ABACOD{#1}}
       \put(2,15)  {\ABACOD{#2}}
       \put(2,10)  {\ABACOD{#3}}
       \put(2,5)   {\ABACOD{#4}}
%      
\end{picture}
    } % \ABACO
    


\cvprfinalcopy % *** Uncomment this line for the final submission
\def\httilde{\mbox{\tt\raisebox{-.5ex}{\symbol{126}}}}
\ifcvprfinal\pagestyle{empty}\fi

\newcommand{\TODO}[1]{TODO: #1}
\newcommand{\CITEONE}[2]{\mbox{#1 \cite{#2}}}
\newcommand{\CITETWO}[3]{\mbox{#1 and #2 \cite{#3}}}
\newcommand{\CITEN}[2]{\mbox{#1 et al. \cite{#2}}}

%%% Paper beginning %%%%%%%%%%%%%%%%%%%%%%%%%%%%%%%%%%%%%%%%%%%%%%%%%%%%%%%%%%%%%%
\begin{document}

%%% Title and authors %%%%%%%%%%%%%%%%%%%%%%%%%%%%%%%%%%%%%%%%%%%%%%%%%%%%%%%%%%%%
\title{Printer ballistics through character's texture analysis}
\author{Adriano Ruggero\thanks{Institute of Computing, University of Campinas (Unicamp). \textbf{Contact}: \tt\small{arruggero@lasca.ic.unicamp.br}}\\
Gabriel Rodrigues\thanks{Institute of Computing, University of Campinas (Unicamp). \textbf{Contact}: \tt\small{gabriel\_rodrigues@aol.com}}\\
Mário Brito\thanks{Institute of Computing, University of Campinas (Unicamp). \textbf{Contact}:
\tt\small{britomar@aedu.com}}\\
Maurício Perez\thanks{Institute of Computing, University of Campinas (Unicamp). \textbf{Contact}:
\tt\small{mauriciolp84@gmail.com}}\\
Anderson Rocha\thanks{Institute of Computing, University of Campinas (Unicamp). \textbf{Contact}: \tt\small{anderson.rocha@ic.unicamp.br}}
}

%%% Abstract %%%%%%%%%%%%%%%%%%%%%%%%%%%%%%%%%%%%%%%%%%%%%%%%%%%%%%%%%%%%%%%%%%%%%
\maketitle
\begin{abstract}
We describe a technique for ballistics of printed documents, that is, link a printed document to a especific printer. The principle of this technique is the analysis of character's texture, extracting some properties from the characters of printed and scanned documents, and relate this properties through a co-occurence matrix. This matrix can be used to create a "fingerprint" of characters (and related printers), that allows identify a specific printer device that print these characters.
\end{abstract}

%%% Introduction %%%%%%%%%%%%%%%%%%%%%%%%%%%%%%%%%%%%%%%%%%%%%%%%%%%%%%%%%%%%%%%%%
\section{Introduction}
In August of 2013, a russian man wrote his own small print in a credit card contract~\cite{RT}. The credit card's administrator bank didn't read the amendments made by the client, and just signed and certified the documents. The changes included unlimited credit line, 0 percent interest rates and no fees. When the bank decided to terminate man's credit card, because overdue payments, he sued they for more than 24 million rubles (US\$ 727.000). How the bank could prove the falsification?

Altough we are living in a digital era, printed documents still are significant part of our daily. Likewise, with the constant reduction in prices and increase in quality of printing equipment, forgeries become increasingly commonplace.

Legal aspects aside, a way to verify if a document, or a part of it, came from a specific device can be character's texture analysis.

Our approach for the analysis of character's texture made up as follows. From printed pages scanned at high resolution, selected characters were extracted. From these characters, we obtained its properties of contrast, correlation, energy and homogeneity, creating with them an co-occurrence matrix. This matrix can be called a ''fingerprint'' of the character. This ''fingerprint'' of  characters is closely related to the printing device which originated it, and can be used to identify which printer was responsible for printing it.

However, slight imperfections may occur during the printing and/or scanning process of documents. To handle these small errors (or variations), characters were selected from different areas of scanned document, their properties were obtained and then classified using machine learning algorithms.


%%% Add section %%%%%%%%%%%%%%%%%%%%%%%%%%%%%%%%%%%%%%%%%%%%%%%%%%%%%%%%%%%%%%%%%%
\section{State-of-the-Art}
Here goes the state-of-the-art research (talk about prior work for solving the same problem). 

%%% Add section %%%%%%%%%%%%%%%%%%%%%%%%%%%%%%%%%%%%%%%%%%%%%%%%%%%%%%%%%%%%%%%%%%
\section{Proposed Solution}
Talk about the proposed solution for the selected problem. 

%%% Add section %%%%%%%%%%%%%%%%%%%%%%%%%%%%%%%%%%%%%%%%%%%%%%%%%%%%%%%%%%%%%%%%%%
\section{Experiments and Discussion}
Talk about the experiments carried out and the obtained results. 

Examples of citations~\cite{Ni_2008, Ni_2009}. For direct citations use something like: 

\CITEONE{Silva}{Silva_2010} for papers with one author.
\CITETWO{Silva}{Souza}{Silva_2010b} for papers with two authors.
\CITEN{Silva}{Silva_2010c} for papers with three or more authors.

Example of a figure of one column. 
\begin{figure}
\begin{center}
	\includegraphics[width=0.99\columnwidth]{example-figure}
	\caption{A figure example spanning one column only.\label{fig:label}}   
\end{center} 
\end{figure}   

Example of a figure spanning two columns. 
\begin{figure*}
\begin{center}
	\includegraphics[width=0.99\textwidth]{example-figure-spanned}
	\caption{A figure example spanning two columns.\label{fig:label2}}   
\end{center} 
\end{figure*}

Example of a table spanning only one column: 

\begin{table}
\begin{center}
\begin{tabular}{l*{6}{c}r}
Team              & P & W & D & L & F  & A & Pts \\
\hline
Manchester United & 6 & 4 & 0 & 2 & 10 & 5 & 12  \\
Celtic            & 6 & 3 & 0 & 3 &  8 & 9 &  9  \\
Benfica           & 6 & 2 & 1 & 3 &  7 & 8 &  7  \\
FC Copenhagen     & 6 & 2 & 1 & 2 &  5 & 8 &  7  \\
\end{tabular}
\end{center}
\end{table}

Example of a table spanning two columns: 

\begin{table*}
\begin{center}
    \begin{tabular}{ | l | l | l | p{8cm} |}
    \hline
    Day & Min Temp & Max Temp & Summary \\ \hline
    Monday & 11C & 22C & A clear day with lots of sunshine.  
    However, the strong breeze will bring down the temperatures. \\ \hline
    Tuesday & 9C & 19C & Cloudy with rain, across many northern regions. Clear spells
    across most of Scotland and Northern Ireland,
    but rain reaching the far northwest. \\ \hline
    Wednesday & 10C & 21C & Rain will still linger for the morning.
    Conditions will improve by early afternoon and continue
    throughout the evening. \\
    \hline
    \end{tabular}
\end{center}    
\end{table*}

%%% Add section %%%%%%%%%%%%%%%%%%%%%%%%%%%%%%%%%%%%%%%%%%%%%%%%%%%%%%%%%%%%%%%%%%
\section{Conclusions and Future Work}
Present the main conclusions of the work as well as some future directions for other people interested in continuing this work. 

%%% References %%%%%%%%%%%%%%%%%%%%%%%%%%%%%%%%%%%%%%%%%%%%%%%%%%%%%%%%%%%%%%%%%%%
{\small
\bibliographystyle{unsrt}
\bibliography{references_printer}
}

\end{document}