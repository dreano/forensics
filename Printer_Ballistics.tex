%%% PREAMBLE - Do not touch %%%%%%%%%%%%%%%%%%%%%%%%%%%%%%%%%%%%%%%%%%%%%%%%%%%%%%
\documentclass[10pt,twocolumn,letterpaper]{article}
%%\usepackage[ansinew]{inputenc}
\usepackage[latin1,utf8]{inputenc}
\usepackage[portuges,brazil,english]{babel}
\usepackage{model}
\usepackage{times}
\usepackage{epsfig}
\usepackage{graphicx}
\usepackage{amsmath}
\usepackage{amssymb}
\usepackage{color}
\usepackage[pagebackref=true,breaklinks=true,letterpaper=true,colorlinks,bookmarks=false]{hyperref}
\input{abaco}

\cvprfinalcopy % *** Uncomment this line for the final submission
\def\httilde{\mbox{\tt\raisebox{-.5ex}{\symbol{126}}}}
\ifcvprfinal\pagestyle{empty}\fi

\newcommand{\TODO}[1]{TODO: #1}
\newcommand{\CITEONE}[2]{\mbox{#1 \cite{#2}}}
\newcommand{\CITETWO}[3]{\mbox{#1 and #2 \cite{#3}}}
\newcommand{\CITEN}[2]{\mbox{#1 et al. \cite{#2}}}

%%% Paper beginning %%%%%%%%%%%%%%%%%%%%%%%%%%%%%%%%%%%%%%%%%%%%%%%%%%%%%%%%%%%%%%
\begin{document}

%%% Title and authors %%%%%%%%%%%%%%%%%%%%%%%%%%%%%%%%%%%%%%%%%%%%%%%%%%%%%%%%%%%%
\title{Printer ballistics through character's texture analysis}
\author{Adriano Ruggero\thanks{Institute of Computing, University of Campinas (Unicamp). \textbf{Contact}: \tt\small{arruggero@lasca.ic.unicamp.br}}\\
Gabriel Rodrigues\thanks{Institute of Computing, University of Campinas (Unicamp). \textbf{Contact}: \tt\small{gabriel\_rodrigues@aol.com}}\\
Mário Brito\thanks{Institute of Computing, University of Campinas (Unicamp). \textbf{Contact}:
\tt\small{britomar@aedu.com}}\\
Maurício Perez\thanks{Institute of Computing, University of Campinas (Unicamp). \textbf{Contact}:
\tt\small{mauriciolp84@gmail.com}}\\
Anderson Rocha\thanks{Institute of Computing, University of Campinas (Unicamp). \textbf{Contact}: \tt\small{anderson.rocha@ic.unicamp.br}}
}

%%% Abstract %%%%%%%%%%%%%%%%%%%%%%%%%%%%%%%%%%%%%%%%%%%%%%%%%%%%%%%%%%%%%%%%%%%%%
\maketitle
\begin{abstract}
We describe a technique for ballistics of printed documents, that is, link a printed document to a especific printer. The principle of this technique is the analysis of character's texture, extracting some properties from the characters of printed and scanned documents, and relate this properties through a co-occurence matrix. This matrix can be used to create a "fingerprint" of characters (and related printers), that allows identify a specific printer device that print these characters.
\end{abstract}

%%% Introduction %%%%%%%%%%%%%%%%%%%%%%%%%%%%%%%%%%%%%%%%%%%%%%%%%%%%%%%%%%%%%%%%%
\section{Introduction}
In August of 2013, a russian man wrote his own small print in a credit card contract~\cite{RT}. The credit card's administrator bank didn't read the amendments made by the client, and just signed and certified the documents. The changes included unlimited credit line, 0 percent interest rates and no fees. When the bank decided to terminate man's credit card, because overdue payments, he sued they for more than 24 million rubles (US\$ 727.000). How the bank could prove the falsification?

Altough we are living in a digital era, printed documents still are significant part of our daily. Likewise, with the constant reduction in prices and increase in quality of printing equipment, forgeries become increasingly commonplace.

Legal aspects aside, a way to verify if a document, or a part of it, came from a specific device can be character's texture analysis.

Our approach for the analysis of character's texture made up as follows. From printed pages scanned at high resolution, selected characters were extracted. From these characters, we obtained its properties of contrast, correlation, energy and homogeneity, creating with them an co-occurrence matrix. This matrix can be called a ''fingerprint'' of the character. This ''fingerprint'' of  characters is closely related to the printing device which originated it, and can be used to identify which printer was responsible for printing it.

However, slight imperfections may occur during the printing and/or scanning process of documents. To handle these small errors (or variations), characters were selected from different areas of scanned document, their properties were obtained and then classified using machine learning algorithms.


%%% Add section %%%%%%%%%%%%%%%%%%%%%%%%%%%%%%%%%%%%%%%%%%%%%%%%%%%%%%%%%%%%%%%%%%
\section{State-of-the-Art}
Here goes the state-of-the-art research (talk about prior work for solving the same problem). 

%%% Add section %%%%%%%%%%%%%%%%%%%%%%%%%%%%%%%%%%%%%%%%%%%%%%%%%%%%%%%%%%%%%%%%%%
\section{Proposed Solution}

Our solution consist in getting the image of characters selected from scanned documents in grayscale, extract its properties of contrast, correlation, energy and homogeneity - creating a co-occurrence matrix - and cluster them by machine learning algorithms.

The characters chosen for this work were "e" and "t", both in lowercase, because they are, respectively, the first and second most common letters in texts written in English~\cite{Letter_Frequency}.

Due to the fact that all scanned documents come from laser printers, it was necessary to take some precautionary measures. Laser printers are known as ''page printers'', while dot matrix printers and inkjet printers are called ''line printers''. This is a crucial difference, and should be considered for more careful study. Line printers print documents line by line from the top of the sheet, keeping a characteristic pattern which periodically repeats for each paper feed. That is, any line printed by this printer model will have basically the same characteristics, regardless of their vertical location in the sheet of paper.

%%% Add section %%%%%%%%%%%%%%%%%%%%%%%%%%%%%%%%%%%%%%%%%%%%%%%%%%%%%%%%%%%%%%%%%%
\section{Experiments and Discussion}

\begin{table}
\begin{center}
\begin{tabular}{l*{2}{c}r}
Printer           & Documents \\
\hline
Brother-HL4070CDW & 28 \\
Canon-D1150 & 28 \\
Canon-MF3240 & 28 \\
Canon-MF4370DN & 28 \\
HP-CLJ-CP2025A & 28 \\
HP-CLJ-CP2025B & 28 \\
HP-JL-CP1518 & 28 \\
Lexmark-E260D & 28 \\
OKI-C330 & 28 \\
Samsung-CLP315 & 28 \\
\end{tabular}
\end{center}
\end{table}


%%% Add section %%%%%%%%%%%%%%%%%%%%%%%%%%%%%%%%%%%%%%%%%%%%%%%%%%%%%%%%%%%%%%%%%%
\section{Conclusions and Future Work}


%%% References %%%%%%%%%%%%%%%%%%%%%%%%%%%%%%%%%%%%%%%%%%%%%%%%%%%%%%%%%%%%%%%%%%%
{\small
\bibliographystyle{unsrt}
\bibliography{references_printer}
}

\end{document}